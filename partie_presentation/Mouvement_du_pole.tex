\section{Passage de la Terre Rigide aux observations}

\begin{frame}
  \frtt{Du modèle de Terre Rigide aux observations}
   REN (\textit{Rigid Earth Nutation}) et observations reliées par une fonction de transfert :
   \begin{align*}
     \frac{\tilde{\eta}_{obs}}{\tilde{\eta}_{REN}}(\sigma) 
     \equiv
     \tilde{T}(\sigma) 
     = \frac{e_R\Omega-\sigma}{e_R\Omega} &\left[ \kappa - \frac{A_f}{A}\gamma\right.- \frac{\Omega(e-\kappa)}{\sigma-\frac{A}{A_m}\Omega(e-\kappa)}\\
     &-\left. \frac{A_f}{A_m} \frac{\Omega (e-\gamma)(e_f-\beta)}{\sigma + \Omega(1+ \frac{A}{A_m}(e_f-\beta))}\right]
   \end{align*}
   	\onslide<2> \textbf{Objectif de la méthodo :}
   	\\ajustement de cette fonction sur les paramètres géométriques (excentricité, déformation...) du noyau. 
\end{frame}
